% Generated by Sphinx.
\def\sphinxdocclass{report}
\newif\ifsphinxKeepOldNames \sphinxKeepOldNamestrue
\documentclass[a4paper,10pt,openany, oneside]{sphinxmanual}
\usepackage{iftex}

\ifPDFTeX
  \usepackage[utf8]{inputenc}
\fi
\ifdefined\DeclareUnicodeCharacter
  \DeclareUnicodeCharacter{00A0}{\nobreakspace}
\fi
\usepackage{cmap}
\usepackage[T1]{fontenc}
\usepackage{amsmath,amssymb,amstext}
\usepackage[english]{babel}
\usepackage{times}
\usepackage[Bjarne]{fncychap}
\usepackage{longtable}
\usepackage{sphinx}
\usepackage{multirow}
\usepackage{eqparbox}


\addto\captionsenglish{\renewcommand{\figurename}{Fig.\@ }}
\addto\captionsenglish{\renewcommand{\tablename}{Table }}
\SetupFloatingEnvironment{literal-block}{name=Listing }

\addto\extrasenglish{\def\pageautorefname{page}}

\setcounter{tocdepth}{1}


\title{pieface Documentation}
\date{Sep 27, 2016}
\release{1.0.0}
\author{James Cumby}
\newcommand{\sphinxlogo}{\sphinxincludegraphics{high_res_icon.png}\par}
\renewcommand{\releasename}{Release}
\makeindex

\makeatletter
\def\PYG@reset{\let\PYG@it=\relax \let\PYG@bf=\relax%
    \let\PYG@ul=\relax \let\PYG@tc=\relax%
    \let\PYG@bc=\relax \let\PYG@ff=\relax}
\def\PYG@tok#1{\csname PYG@tok@#1\endcsname}
\def\PYG@toks#1+{\ifx\relax#1\empty\else%
    \PYG@tok{#1}\expandafter\PYG@toks\fi}
\def\PYG@do#1{\PYG@bc{\PYG@tc{\PYG@ul{%
    \PYG@it{\PYG@bf{\PYG@ff{#1}}}}}}}
\def\PYG#1#2{\PYG@reset\PYG@toks#1+\relax+\PYG@do{#2}}

\expandafter\def\csname PYG@tok@gd\endcsname{\def\PYG@tc##1{\textcolor[rgb]{0.63,0.00,0.00}{##1}}}
\expandafter\def\csname PYG@tok@gu\endcsname{\let\PYG@bf=\textbf\def\PYG@tc##1{\textcolor[rgb]{0.50,0.00,0.50}{##1}}}
\expandafter\def\csname PYG@tok@gt\endcsname{\def\PYG@tc##1{\textcolor[rgb]{0.00,0.27,0.87}{##1}}}
\expandafter\def\csname PYG@tok@gs\endcsname{\let\PYG@bf=\textbf}
\expandafter\def\csname PYG@tok@gr\endcsname{\def\PYG@tc##1{\textcolor[rgb]{1.00,0.00,0.00}{##1}}}
\expandafter\def\csname PYG@tok@cm\endcsname{\let\PYG@it=\textit\def\PYG@tc##1{\textcolor[rgb]{0.25,0.50,0.56}{##1}}}
\expandafter\def\csname PYG@tok@vg\endcsname{\def\PYG@tc##1{\textcolor[rgb]{0.73,0.38,0.84}{##1}}}
\expandafter\def\csname PYG@tok@vi\endcsname{\def\PYG@tc##1{\textcolor[rgb]{0.73,0.38,0.84}{##1}}}
\expandafter\def\csname PYG@tok@mh\endcsname{\def\PYG@tc##1{\textcolor[rgb]{0.13,0.50,0.31}{##1}}}
\expandafter\def\csname PYG@tok@cs\endcsname{\def\PYG@tc##1{\textcolor[rgb]{0.25,0.50,0.56}{##1}}\def\PYG@bc##1{\setlength{\fboxsep}{0pt}\colorbox[rgb]{1.00,0.94,0.94}{\strut ##1}}}
\expandafter\def\csname PYG@tok@ge\endcsname{\let\PYG@it=\textit}
\expandafter\def\csname PYG@tok@vc\endcsname{\def\PYG@tc##1{\textcolor[rgb]{0.73,0.38,0.84}{##1}}}
\expandafter\def\csname PYG@tok@il\endcsname{\def\PYG@tc##1{\textcolor[rgb]{0.13,0.50,0.31}{##1}}}
\expandafter\def\csname PYG@tok@go\endcsname{\def\PYG@tc##1{\textcolor[rgb]{0.20,0.20,0.20}{##1}}}
\expandafter\def\csname PYG@tok@cp\endcsname{\def\PYG@tc##1{\textcolor[rgb]{0.00,0.44,0.13}{##1}}}
\expandafter\def\csname PYG@tok@gi\endcsname{\def\PYG@tc##1{\textcolor[rgb]{0.00,0.63,0.00}{##1}}}
\expandafter\def\csname PYG@tok@gh\endcsname{\let\PYG@bf=\textbf\def\PYG@tc##1{\textcolor[rgb]{0.00,0.00,0.50}{##1}}}
\expandafter\def\csname PYG@tok@ni\endcsname{\let\PYG@bf=\textbf\def\PYG@tc##1{\textcolor[rgb]{0.84,0.33,0.22}{##1}}}
\expandafter\def\csname PYG@tok@nl\endcsname{\let\PYG@bf=\textbf\def\PYG@tc##1{\textcolor[rgb]{0.00,0.13,0.44}{##1}}}
\expandafter\def\csname PYG@tok@nn\endcsname{\let\PYG@bf=\textbf\def\PYG@tc##1{\textcolor[rgb]{0.05,0.52,0.71}{##1}}}
\expandafter\def\csname PYG@tok@no\endcsname{\def\PYG@tc##1{\textcolor[rgb]{0.38,0.68,0.84}{##1}}}
\expandafter\def\csname PYG@tok@na\endcsname{\def\PYG@tc##1{\textcolor[rgb]{0.25,0.44,0.63}{##1}}}
\expandafter\def\csname PYG@tok@nb\endcsname{\def\PYG@tc##1{\textcolor[rgb]{0.00,0.44,0.13}{##1}}}
\expandafter\def\csname PYG@tok@nc\endcsname{\let\PYG@bf=\textbf\def\PYG@tc##1{\textcolor[rgb]{0.05,0.52,0.71}{##1}}}
\expandafter\def\csname PYG@tok@nd\endcsname{\let\PYG@bf=\textbf\def\PYG@tc##1{\textcolor[rgb]{0.33,0.33,0.33}{##1}}}
\expandafter\def\csname PYG@tok@ne\endcsname{\def\PYG@tc##1{\textcolor[rgb]{0.00,0.44,0.13}{##1}}}
\expandafter\def\csname PYG@tok@nf\endcsname{\def\PYG@tc##1{\textcolor[rgb]{0.02,0.16,0.49}{##1}}}
\expandafter\def\csname PYG@tok@si\endcsname{\let\PYG@it=\textit\def\PYG@tc##1{\textcolor[rgb]{0.44,0.63,0.82}{##1}}}
\expandafter\def\csname PYG@tok@s2\endcsname{\def\PYG@tc##1{\textcolor[rgb]{0.25,0.44,0.63}{##1}}}
\expandafter\def\csname PYG@tok@nt\endcsname{\let\PYG@bf=\textbf\def\PYG@tc##1{\textcolor[rgb]{0.02,0.16,0.45}{##1}}}
\expandafter\def\csname PYG@tok@nv\endcsname{\def\PYG@tc##1{\textcolor[rgb]{0.73,0.38,0.84}{##1}}}
\expandafter\def\csname PYG@tok@s1\endcsname{\def\PYG@tc##1{\textcolor[rgb]{0.25,0.44,0.63}{##1}}}
\expandafter\def\csname PYG@tok@ch\endcsname{\let\PYG@it=\textit\def\PYG@tc##1{\textcolor[rgb]{0.25,0.50,0.56}{##1}}}
\expandafter\def\csname PYG@tok@m\endcsname{\def\PYG@tc##1{\textcolor[rgb]{0.13,0.50,0.31}{##1}}}
\expandafter\def\csname PYG@tok@gp\endcsname{\let\PYG@bf=\textbf\def\PYG@tc##1{\textcolor[rgb]{0.78,0.36,0.04}{##1}}}
\expandafter\def\csname PYG@tok@sh\endcsname{\def\PYG@tc##1{\textcolor[rgb]{0.25,0.44,0.63}{##1}}}
\expandafter\def\csname PYG@tok@ow\endcsname{\let\PYG@bf=\textbf\def\PYG@tc##1{\textcolor[rgb]{0.00,0.44,0.13}{##1}}}
\expandafter\def\csname PYG@tok@sx\endcsname{\def\PYG@tc##1{\textcolor[rgb]{0.78,0.36,0.04}{##1}}}
\expandafter\def\csname PYG@tok@bp\endcsname{\def\PYG@tc##1{\textcolor[rgb]{0.00,0.44,0.13}{##1}}}
\expandafter\def\csname PYG@tok@c1\endcsname{\let\PYG@it=\textit\def\PYG@tc##1{\textcolor[rgb]{0.25,0.50,0.56}{##1}}}
\expandafter\def\csname PYG@tok@o\endcsname{\def\PYG@tc##1{\textcolor[rgb]{0.40,0.40,0.40}{##1}}}
\expandafter\def\csname PYG@tok@kc\endcsname{\let\PYG@bf=\textbf\def\PYG@tc##1{\textcolor[rgb]{0.00,0.44,0.13}{##1}}}
\expandafter\def\csname PYG@tok@c\endcsname{\let\PYG@it=\textit\def\PYG@tc##1{\textcolor[rgb]{0.25,0.50,0.56}{##1}}}
\expandafter\def\csname PYG@tok@mf\endcsname{\def\PYG@tc##1{\textcolor[rgb]{0.13,0.50,0.31}{##1}}}
\expandafter\def\csname PYG@tok@err\endcsname{\def\PYG@bc##1{\setlength{\fboxsep}{0pt}\fcolorbox[rgb]{1.00,0.00,0.00}{1,1,1}{\strut ##1}}}
\expandafter\def\csname PYG@tok@mb\endcsname{\def\PYG@tc##1{\textcolor[rgb]{0.13,0.50,0.31}{##1}}}
\expandafter\def\csname PYG@tok@ss\endcsname{\def\PYG@tc##1{\textcolor[rgb]{0.32,0.47,0.09}{##1}}}
\expandafter\def\csname PYG@tok@sr\endcsname{\def\PYG@tc##1{\textcolor[rgb]{0.14,0.33,0.53}{##1}}}
\expandafter\def\csname PYG@tok@mo\endcsname{\def\PYG@tc##1{\textcolor[rgb]{0.13,0.50,0.31}{##1}}}
\expandafter\def\csname PYG@tok@kd\endcsname{\let\PYG@bf=\textbf\def\PYG@tc##1{\textcolor[rgb]{0.00,0.44,0.13}{##1}}}
\expandafter\def\csname PYG@tok@mi\endcsname{\def\PYG@tc##1{\textcolor[rgb]{0.13,0.50,0.31}{##1}}}
\expandafter\def\csname PYG@tok@kn\endcsname{\let\PYG@bf=\textbf\def\PYG@tc##1{\textcolor[rgb]{0.00,0.44,0.13}{##1}}}
\expandafter\def\csname PYG@tok@cpf\endcsname{\let\PYG@it=\textit\def\PYG@tc##1{\textcolor[rgb]{0.25,0.50,0.56}{##1}}}
\expandafter\def\csname PYG@tok@kr\endcsname{\let\PYG@bf=\textbf\def\PYG@tc##1{\textcolor[rgb]{0.00,0.44,0.13}{##1}}}
\expandafter\def\csname PYG@tok@s\endcsname{\def\PYG@tc##1{\textcolor[rgb]{0.25,0.44,0.63}{##1}}}
\expandafter\def\csname PYG@tok@kp\endcsname{\def\PYG@tc##1{\textcolor[rgb]{0.00,0.44,0.13}{##1}}}
\expandafter\def\csname PYG@tok@w\endcsname{\def\PYG@tc##1{\textcolor[rgb]{0.73,0.73,0.73}{##1}}}
\expandafter\def\csname PYG@tok@kt\endcsname{\def\PYG@tc##1{\textcolor[rgb]{0.56,0.13,0.00}{##1}}}
\expandafter\def\csname PYG@tok@sc\endcsname{\def\PYG@tc##1{\textcolor[rgb]{0.25,0.44,0.63}{##1}}}
\expandafter\def\csname PYG@tok@sb\endcsname{\def\PYG@tc##1{\textcolor[rgb]{0.25,0.44,0.63}{##1}}}
\expandafter\def\csname PYG@tok@k\endcsname{\let\PYG@bf=\textbf\def\PYG@tc##1{\textcolor[rgb]{0.00,0.44,0.13}{##1}}}
\expandafter\def\csname PYG@tok@se\endcsname{\let\PYG@bf=\textbf\def\PYG@tc##1{\textcolor[rgb]{0.25,0.44,0.63}{##1}}}
\expandafter\def\csname PYG@tok@sd\endcsname{\let\PYG@it=\textit\def\PYG@tc##1{\textcolor[rgb]{0.25,0.44,0.63}{##1}}}

\def\PYGZbs{\char`\\}
\def\PYGZus{\char`\_}
\def\PYGZob{\char`\{}
\def\PYGZcb{\char`\}}
\def\PYGZca{\char`\^}
\def\PYGZam{\char`\&}
\def\PYGZlt{\char`\<}
\def\PYGZgt{\char`\>}
\def\PYGZsh{\char`\#}
\def\PYGZpc{\char`\%}
\def\PYGZdl{\char`\$}
\def\PYGZhy{\char`\-}
\def\PYGZsq{\char`\'}
\def\PYGZdq{\char`\"}
\def\PYGZti{\char`\~}
% for compatibility with earlier versions
\def\PYGZat{@}
\def\PYGZlb{[}
\def\PYGZrb{]}
\makeatother

\renewcommand\PYGZsq{\textquotesingle}

\begin{document}

\maketitle
\tableofcontents
\phantomsection\label{index::doc}


\href{http://www.csec.ed.ac.uk}{pieface} is a Python program to fit distortions in atomic coordination polyhedra.

pieface is designed to be easy to use, versatile and easily extendable. The {\hyperref[glossary:term\string-mbe]{\sphinxtermref{\DUrole{xref,std,std-term}{MBE}}}} ellipsoid method used
to fit polyhedra is very general, and can be applied to a wide range of coordination problems. Full details of the
method, and a few interesting examples can be seen in the original research publication:
\href{http://www.csec.ed.ac.uk}{James Cumby \& J. Paul Attfield, \emph{Ellipsoidal Analysis of Coordination Polyhedra}.}

To get started with pieface, have a look at the {\hyperref[introduction:introduction]{\sphinxcrossref{\DUrole{std,std-ref}{Introduction}}}} and {\hyperref[tutorial:tutorials]{\sphinxcrossref{\DUrole{std,std-ref}{Tutorials}}}}. Further documentation
can be found below.


\chapter{Introduction to PIEFACE}
\label{introduction:introduction}\label{introduction:introduction-to-pieface}\label{introduction::doc}\label{introduction:pieface-fitting-distorted-coordination-polyhedra}

\section{Polyhedra Inscribing Ellipsoids For Analysis of Coordination Environments}
\label{introduction:polyhedra-inscribing-ellipsoids-for-analysis-of-coordination-environments}
\textbf{P}olyhedra \textbf{I}nscribing \textbf{E}llipsoids \textbf{F}or \textbf{A}nalysis of \textbf{C}oordination \textbf{E}nvironments (or \href{http://www.csec.ed.ac.uk}{pieface}) is an open source \href{https://www.python.org/\textgreater{}}{Python} project intended for the
analysis of distortions of a chemical coordination polyhedron. The analysis is very general way, irrespective of polyhedron size or nature of the distortion, and could be applied
to any problem where a coordination sphere with known coordinates exists, from extended inorganic solids to organic molecules.
Full details of the method can be found in the original research article, \href{http://www.csec.ed.ac.uk}{Ellipsoidal Analysis of Coordination Polyhedra}, including some interesting examples.

For many crystallographic polyhedra, distortion is difficult to rationalise simply in terms of deviations of bond lengths or bond angles from \emph{ideal} values. By fitting the smallest volume {\hyperref[glossary:term\string-ellipsoid]{\sphinxtermref{\DUrole{xref,std,std-term}{ellipsoid}}}}
around the polyhedron, distortions are defined in terms of the three principal axes of the ellipsoid and it's orientation in space. The distortion of this {\hyperref[glossary:term\string-ellipsoid]{\sphinxtermref{\DUrole{xref,std,std-term}{ellipsoid}}}} can then give a simple description
of the distortions involved.


\section{Citing pieface}
\label{introduction:citing-pieface}
If you use pieface, please \textbf{cite} it:
\begin{quote}

\href{http://www.csec.ed.ac.uk}{James Cumby \& J. Paul Attfield, \emph{Ellipsoidal Analysis of Coordination Polyhedra}.}
\end{quote}


\section{Getting Started}
\label{introduction:getting-started}
Once installed (see {\hyperref[installation:installation]{\sphinxcrossref{\DUrole{std,std-ref}{Installation}}}}) pieface can be accessed either through a command-line interface (\sphinxcode{CIFellipsoid}) or a user-friendly graphical interface (\sphinxcode{distellipsoidGUI}).
Both should be available on the system command line/terminal, and also from the start menu (on Windows).
\begin{description}
\item[{\sphinxcode{distellipsoidGUI}}] \leavevmode
Should be adequate for most users. This {\hyperref[glossary:term\string-gui]{\sphinxtermref{\DUrole{xref,std,std-term}{GUI}}}} provides a clickable interface to commonly used \href{http://www.csec.ed.ac.uk}{pieface} functions, and allows users to
import {\hyperref[glossary:term\string-cif]{\sphinxtermref{\DUrole{xref,std,std-term}{CIF}}}} files for analysis, and examine/save the resulting output.

\item[{\sphinxcode{CIFellipsoid}}] \leavevmode
Gives terminal-based access to a wider range of capabilities, details of which can be found by typing \sphinxcode{CIFellipsoid} \sphinxcode{-{-}help}.

\end{description}

In both cases, the input required is one or more {\hyperref[glossary:term\string-cif]{\sphinxtermref{\DUrole{xref,std,std-term}{CIF}}}} files, and a list of atom types or labels to be used as polyhedron centres and ligands.
Once calculated (which can take some time for a large number of files) the resulting ellipsoid parameters are saved as a text file (one per {\hyperref[glossary:term\string-cif]{\sphinxtermref{\DUrole{xref,std,std-term}{CIF}}}} file).
Using \sphinxcode{distellipsoidGUI} the resulting ellipsoids and parameters can also be visualised interactively.

More detailed examples of usage can be found in {\hyperref[tutorial:tutorials]{\sphinxcrossref{\DUrole{std,std-ref}{Tutorials}}}}.


\section{License}
\label{introduction:license}
pieface is distributed under the \href{https://opensource.org/licenses/MIT}{MIT license} (see \DUrole{xref,doc}{\textless{}../license.txt\textgreater{}}). Any use of the software should be cited


\section{Disclaimer}
\label{introduction:disclaimer}
This software is provided as-is, on a best-effort basis. The authors accept no liabilities associated with the use of this software.
It has been tested for accuracy of results for a number of cases, but only for uses that the authors can think of. We would be interested
to hear of any suggestions for new uses, or potential additions to the software.

We will attempt to correct any bugs as they are found on a best-effort basis!


\section{Authors}
\label{introduction:authors}
James Cumby - james.cumby ``at'' ed.ac.uk


\chapter{Installation}
\label{installation:installation}\label{installation:citation}\label{installation::doc}\label{installation:id1}
\href{http://www.csec.ed.ac.uk}{pieface} is written in pure \href{https://www.python.org/\textgreater{}}{Python}. While this makes it highly transferrable between operating systems,
it does require a number of other Python packages to operate.


\section{Requirements}
\label{installation:requirements}\begin{itemize}
\item {} 
\href{https://www.python.org/}{Python 2.7} (currently NOT Python 3)

\item {} 
\href{http://www.numpy.org}{NumPy} (at least version 1.9)

\item {} 
\href{http://matplotlib.org/}{matplotlib} (1.4.3 or higher)

\item {} 
\href{https://bitbucket.org/jamesrhester/pycifrw/overview}{PyCifRW} (3.3 or higher)

\item {} 
\href{https://docs.python.org/2/library/multiprocessing.html}{multiprocessing} (2.6.2 or higher)

\item {} 
\href{http://pandas.pydata.org/}{pandas} (0.17 or higher)

\end{itemize}


\section{Installing}
\label{installation:installing}
Detailed installation instructions specific to different operating systems can be found under {\hyperref[installation:windows]{\sphinxcrossref{\DUrole{std,std-ref}{Windows}}}}, {\hyperref[installation:macosx]{\sphinxcrossref{\DUrole{std,std-ref}{MAC OS X}}}} and {\hyperref[installation:linux]{\sphinxcrossref{\DUrole{std,std-ref}{Linux derivatives}}}}.

pieface is registered on \href{https://pypi.python.org/pypi}{PyPI}, therefore if you already have a working Python distribution, installation may be
as simple as:

\begin{Verbatim}[commandchars=\\\{\}]
\PYG{n}{pip} \PYG{n}{install} \PYG{n}{pieface}
\end{Verbatim}

or alternatively by manually downloading and installing:
\begin{itemize}
\item {} 
Download \sphinxcode{pieface.tar.gz} to your computer

\item {} 
Unpack to a local directory

\item {} 
Install using:

\begin{Verbatim}[commandchars=\\\{\}]
\PYG{n}{python} \PYG{n}{setup}\PYG{o}{.}\PYG{n}{py} \PYG{n}{install}
\end{Verbatim}

\end{itemize}

In reality, installation can sometimes be operating-system specific.


\subsection{Windows}
\label{installation:windows}\label{installation:id2}
Due to problems with ensuring correct dependencies, the recommended method for obtaining pieface for Windows is to download the self-contained installer
WinSetup\_PIEFACE\_1.0.0.exe and run it, following the on-screen prompts. This will also (optionally) add pieface shortcuts to the Start Menu and Windows Desktop,
as well as making the two main scripts accessible from the Windows Command Line.

The installer comes packaged with a minimal Python runtime environment, therefore this installer will work without (and not interfere with an existing) Python
installation.


\subsection{MAC OS X}
\label{installation:macosx}\label{installation:mac-os-x}
Unfortunately pieface is not currently available as a pre-built MAC distribution, as the author does not have access to that operating system!

Installing using the simple \sphinxcode{pip} or \sphinxcode{python setup.py install} routes may be possible using the default Python environment...


\subsection{Linux derivatives}
\label{installation:linux-derivatives}\label{installation:linux}
Unix-like operating systems generally come with python included. In this case,:

\begin{Verbatim}[commandchars=\\\{\}]
\PYG{n}{pip} \PYG{n}{install} \PYG{n}{pieface}
\end{Verbatim}

should work as expected.


\section{Installation from Sources}
\label{installation:installation-from-sources}

\subsection{Stable Build}
\label{installation:stable-build}
pieface can also be installed from the source distribution. The current stable build is \href{http://www.csec.ed.ac.uk}{pieface\_1.0.0.tar.gz}.
Once downloaded, this file should be unpacked into the desired directory (\sphinxcode{tar -xzf pieface\_1.0.0.tar.gz}) before following the {\hyperref[installation:setup]{\sphinxcrossref{\DUrole{std,std-ref}{setup instructions}}}}.


\subsection{Development Version}
\label{installation:development-version}
The latest development version of pieface can be obtained from the \href{http://www.github.org}{pieface repository} using GIT:

\begin{Verbatim}[commandchars=\\\{\}]
\PYG{n}{GIT} \PYG{n}{clone} \PYG{n}{git}\PYG{p}{:}\PYG{o}{/}\PYG{o}{/}\PYG{n}{github}\PYG{o}{.}\PYG{n}{com}\PYG{o}{/}\PYG{n}{JCumby}\PYG{o}{/}\PYG{n}{pieface} \PYG{o}{.}
\end{Verbatim}

To update the repository at a later date, use:

\begin{Verbatim}[commandchars=\\\{\}]
\PYG{n}{GIT} \PYG{n}{pull}
\end{Verbatim}

In both cases, you should then change into the resulting directory, and follow the {\hyperref[installation:setup]{\sphinxcrossref{\DUrole{std,std-ref}{setup instructions}}}}.


\subsection{Manual Install}
\label{installation:setup}\label{installation:manual-install}
Once the source code has been downloaded, it is then necessary to install it using Python from within the
main pieface directory:

\begin{Verbatim}[commandchars=\\\{\}]
\PYG{n}{Python} \PYG{n}{setup}\PYG{o}{.}\PYG{n}{py} \PYG{n}{install}
\end{Verbatim}

This may require all dependencies to already be installed.


\section{Testing}
\label{installation:testing}
The package contains some basic unit tests, which can be run from within the main pieface directory with the command:

\begin{Verbatim}[commandchars=\\\{\}]
\PYG{n}{python} \PYG{n}{setup}\PYG{o}{.}\PYG{n}{py} \PYG{n}{test}
\end{Verbatim}

All tests should pass without exceptions - if not, please send me a bug report.


\section{Run It!}
\label{installation:run-it}
Once correctly installed, the easiest way to access pieface is using either \sphinxcode{distellipsoidGUI} or \sphinxcode{CIFellipsoid} (see {\hyperref[tutorial:tutorials]{\sphinxcrossref{\DUrole{std,std-ref}{Tutorials}}}}).


\chapter{Tutorials}
\label{tutorial:tutorials}\label{tutorial:citation}\label{tutorial::doc}\label{tutorial:id1}
Still under construction!


\chapter{User-interface Documentation}
\label{script_input:citation}\label{script_input::doc}\label{script_input:user-interface-documentation}

\section{\texttt{CIFellipsoid}}
\label{script_input:cmdprog}
This is the main command line script for using pieface to compute {\hyperref[glossary:term\string-mbe]{\sphinxtermref{\DUrole{xref,std,std-term}{MBE}}}} from one or more {\hyperref[glossary:term\string-cif]{\sphinxtermref{\DUrole{xref,std,std-term}{CIF}}}} files.
Details of the input parameters are given below, or by typing \sphinxcode{CIFellipsoid} \sphinxcode{-{-}help}.
\index{CIFellipsoid command line option!cifs}\index{cifs!CIFellipsoid command line option}

\begin{fulllineitems}
\phantomsection\label{script_input:cmdoption-CIFellipsoid-arg-cifs}\pysigline{\sphinxbfcode{cifs}\sphinxcode{}}
Name(s) of CIF files to import as a space-delimited list. Can also accept valid web address(es).

\end{fulllineitems}

\index{CIFellipsoid command line option!-m, --metal}\index{-m, --metal!CIFellipsoid command line option}

\begin{fulllineitems}
\phantomsection\label{script_input:cmdoption-CIFellipsoid-m}\pysigline{\sphinxbfcode{-m}\sphinxcode{}\sphinxcode{,~}\sphinxbfcode{-{-}metal}\sphinxcode{}}~\begin{description}
\item[{Site label(s) (as found in the CIF file) of polyhedron centres to analyse (e.g. Fe1).}] \leavevmode\begin{description}
\item[{\emph{Most} \href{https://docs.python.org/2/library/re.html}{regular expressions} can be used to make searching easier:}] \leavevmode\begin{quote}\begin{description}
\item[{\sphinxcode{Al*}}] \leavevmode
matches any site label starting \sphinxcode{Al} (\sphinxcode{Al1}, \sphinxcode{Al2} ... \sphinxcode{Al9999} etc.)

\item[{\sphinxcode{Al?}}] \leavevmode
matches any label beginning \sphinxtitleref{Al}, but only 3 characters in length (e.g. \sphinxcode{Al1} - \sphinxcode{Al9})

\item[{\sphinxcode{Al{[}1-9{]}}}] \leavevmode
matches any site \sphinxcode{Al1} - \sphinxcode{Al9}

\end{description}\end{quote}

\item[{In addition to most normal regular expressions, preceding any label by \sphinxcode{\#} will omit it from the search:}] \leavevmode\begin{quote}\begin{description}
\item[{\sphinxcode{\#Al1}}] \leavevmode
will omit \sphinxcode{Al1} from the list of acceptable centres.

\end{description}\end{quote}

\end{description}

By combining these terms, it should be possible to specify most desired combinations.

\end{description}

\end{fulllineitems}

\index{CIFellipsoid command line option!-o, --output}\index{-o, --output!CIFellipsoid command line option}

\begin{fulllineitems}
\phantomsection\label{script_input:cmdoption-CIFellipsoid-o}\pysigline{\sphinxbfcode{-o}\sphinxcode{}\sphinxcode{,~}\sphinxbfcode{-{-}output}\sphinxcode{}}
Name(s) of output files to save results to. Default is \textless{}CIF Name\textgreater{}.txt.

\end{fulllineitems}

\index{CIFellipsoid command line option!-r, --radius}\index{-r, --radius!CIFellipsoid command line option}

\begin{fulllineitems}
\phantomsection\label{script_input:cmdoption-CIFellipsoid-r}\pysigline{\sphinxbfcode{-r}\sphinxcode{}\sphinxcode{,~}\sphinxbfcode{-{-}radius}\sphinxcode{}}
Maximum distance to treat a ligand as part of a polyhedron (default 3 Angstrom).

\end{fulllineitems}

\index{CIFellipsoid command line option!-l, --ligandtypes}\index{-l, --ligandtypes!CIFellipsoid command line option}

\begin{fulllineitems}
\phantomsection\label{script_input:cmdoption-CIFellipsoid-l}\pysigline{\sphinxbfcode{-l}\sphinxcode{}\sphinxcode{,~}\sphinxbfcode{-{-}ligandtypes}\sphinxcode{}}
Types of atom (as specified by CIF atom\_type) to be considered as valid ligands (can use regular expressions).
Default is all  atom types.

\end{fulllineitems}

\index{CIFellipsoid command line option!-n, --ligandnames}\index{-n, --ligandnames!CIFellipsoid command line option}

\begin{fulllineitems}
\phantomsection\label{script_input:cmdoption-CIFellipsoid-n}\pysigline{\sphinxbfcode{-n}\sphinxcode{}\sphinxcode{,~}\sphinxbfcode{-{-}ligandnames}\sphinxcode{}}
Atom labels to use as ligands (same syntax as --metal). By default, any ligand allowed by --ligandtypes is allowed.
The combination of --ligandtypes and --ligandnames is taken as an \sphinxcode{AND}-like operation, such that sites are only
excluded if done so explicitly.

\end{fulllineitems}

\index{CIFellipsoid command line option!-t, --tolerance}\index{-t, --tolerance!CIFellipsoid command line option}

\begin{fulllineitems}
\phantomsection\label{script_input:cmdoption-CIFellipsoid-t}\pysigline{\sphinxbfcode{-t}\sphinxcode{}\sphinxcode{,~}\sphinxbfcode{-{-}tolerance}\sphinxcode{}}
Tolerance to use for fitting ellipsoid to points (default 1e-6).

\end{fulllineitems}

\index{CIFellipsoid command line option!--maxcycles}\index{--maxcycles!CIFellipsoid command line option}

\begin{fulllineitems}
\phantomsection\label{script_input:cmdoption-CIFellipsoid--maxcycles}\pysigline{\sphinxbfcode{-{-}maxcycles}\sphinxcode{}}
Maximum number of iterations to perform for fitting (default infinite).

\end{fulllineitems}

\index{CIFellipsoid command line option!-N}\index{-N!CIFellipsoid command line option}

\begin{fulllineitems}
\phantomsection\label{script_input:cmdoption-CIFellipsoid-N}\pysigline{\sphinxbfcode{-N}\sphinxcode{}}
Don't save results to text files

\end{fulllineitems}

\index{CIFellipsoid command line option!-W, --overwriteall}\index{-W, --overwriteall!CIFellipsoid command line option}

\begin{fulllineitems}
\phantomsection\label{script_input:cmdoption-CIFellipsoid-W}\pysigline{\sphinxbfcode{-W}\sphinxcode{}\sphinxcode{,~}\sphinxbfcode{-{-}overwriteall}\sphinxcode{}}
If existing results files already exist, force pieface to overwrite them all.

\end{fulllineitems}

\index{CIFellipsoid command line option!-P, --PrintLabels}\index{-P, --PrintLabels!CIFellipsoid command line option}

\begin{fulllineitems}
\phantomsection\label{script_input:cmdoption-CIFellipsoid-P}\pysigline{\sphinxbfcode{-P}\sphinxcode{}\sphinxcode{,~}\sphinxbfcode{-{-}PrintLabels}\sphinxcode{}}
Print all valid site labels for each CIF file supplied.

\end{fulllineitems}

\index{CIFellipsoid command line option!-U, --Unthreaded}\index{-U, --Unthreaded!CIFellipsoid command line option}

\begin{fulllineitems}
\phantomsection\label{script_input:cmdoption-CIFellipsoid-U}\pysigline{\sphinxbfcode{-U}\sphinxcode{}\sphinxcode{,~}\sphinxbfcode{-{-}Unthreaded}\sphinxcode{}}
Turn off parallel processing of CIF files.

\end{fulllineitems}

\index{CIFellipsoid command line option!--procs}\index{--procs!CIFellipsoid command line option}

\begin{fulllineitems}
\phantomsection\label{script_input:cmdoption-CIFellipsoid--procs}\pysigline{\sphinxbfcode{-{-}procs}\sphinxcode{}}
Number of processors to use for parallel processing (default all).

\end{fulllineitems}

\index{CIFellipsoid command line option!--noplot}\index{--noplot!CIFellipsoid command line option}

\begin{fulllineitems}
\phantomsection\label{script_input:cmdoption-CIFellipsoid--noplot}\pysigline{\sphinxbfcode{-{-}noplot}\sphinxcode{}}
Don't produce interactive ellipsoid images after calculation.

\end{fulllineitems}

\index{CIFellipsoid command line option!--writelog}\index{--writelog!CIFellipsoid command line option}

\begin{fulllineitems}
\phantomsection\label{script_input:cmdoption-CIFellipsoid--writelog}\pysigline{\sphinxbfcode{-{-}writelog}\sphinxcode{}}
Write a debugging log to \sphinxcode{debug.log}.

\end{fulllineitems}



\chapter{API Reference}
\label{api_reference:api}\label{api_reference:api-reference}\label{api_reference:citation}\label{api_reference::doc}

\section{Ellipsoid Module}
\label{api_reference:ellipsoid-module}
Contains the core functionality of pieface, responsible for fitting a {\hyperref[glossary:term\string-mbe]{\sphinxtermref{\DUrole{xref,std,std-term}{MBE}}}} to a set of points in Cartesian space.
\phantomsection\label{api_reference:module-distellipsoid.ellipsoid}\index{distellipsoid.ellipsoid (module)}
Module containing functions for calculating a minimum bounding ellipsoid from a set of points.

Contains a class to hold Ellipsoid object/properties, and functions to compute
the ellipsoid from a set of points
\begin{description}
\item[{Basic usage:}] \leavevmode\begin{itemize}
\item {} 
Set up an Ellipsoid object

\item {} 
Assign a set of (cartesian) points to be used for fitting

\item {} 
Use method findellipsoid() to compute minimum bounding ellipsoid

\item {} 
After that, all other properties should be available.

\end{itemize}

\end{description}
\index{Ellipsoid (class in distellipsoid.ellipsoid)}

\begin{fulllineitems}
\phantomsection\label{api_reference:distellipsoid.ellipsoid.Ellipsoid}\pysiglinewithargsret{\sphinxstrong{class }\sphinxcode{distellipsoid.ellipsoid.}\sphinxbfcode{Ellipsoid}}{\emph{points=None}, \emph{tolerance=1e-06}}{}
An object for computing various hyperellipse properties.
\index{centreaxes() (distellipsoid.ellipsoid.Ellipsoid method)}

\begin{fulllineitems}
\phantomsection\label{api_reference:distellipsoid.ellipsoid.Ellipsoid.centreaxes}\pysiglinewithargsret{\sphinxbfcode{centreaxes}}{}{}
Return displacement along ellipsoid axes.

\end{fulllineitems}

\index{centredisp() (distellipsoid.ellipsoid.Ellipsoid method)}

\begin{fulllineitems}
\phantomsection\label{api_reference:distellipsoid.ellipsoid.Ellipsoid.centredisp}\pysiglinewithargsret{\sphinxbfcode{centredisp}}{}{}
Return total displacement of centre.

\end{fulllineitems}

\index{ellipsvol() (distellipsoid.ellipsoid.Ellipsoid method)}

\begin{fulllineitems}
\phantomsection\label{api_reference:distellipsoid.ellipsoid.Ellipsoid.ellipsvol}\pysiglinewithargsret{\sphinxbfcode{ellipsvol}}{}{}
Return volume of ellipsoid.

\end{fulllineitems}

\index{findellipsoid() (distellipsoid.ellipsoid.Ellipsoid method)}

\begin{fulllineitems}
\phantomsection\label{api_reference:distellipsoid.ellipsoid.Ellipsoid.findellipsoid}\pysiglinewithargsret{\sphinxbfcode{findellipsoid}}{\emph{suppliedpts=None}, \emph{**kwargs}}{}
Determine the number of dimensions required for hyperellipse, and call then compute it with getminvol.

\end{fulllineitems}

\index{getminvol() (distellipsoid.ellipsoid.Ellipsoid method)}

\begin{fulllineitems}
\phantomsection\label{api_reference:distellipsoid.ellipsoid.Ellipsoid.getminvol}\pysiglinewithargsret{\sphinxbfcode{getminvol}}{\emph{points=None}, \emph{maxcycles=None}}{}
Find the minimum bounding ellipsoid for a set of points using the Khachiyan algorithm.

This can be quite time-consuming if a small tolerance is required, and ellipsoid axes
lie a long way from axis directions.

\end{fulllineitems}

\index{meanrad() (distellipsoid.ellipsoid.Ellipsoid method)}

\begin{fulllineitems}
\phantomsection\label{api_reference:distellipsoid.ellipsoid.Ellipsoid.meanrad}\pysiglinewithargsret{\sphinxbfcode{meanrad}}{}{}
Return the mean radius.

\end{fulllineitems}

\index{numpoints() (distellipsoid.ellipsoid.Ellipsoid method)}

\begin{fulllineitems}
\phantomsection\label{api_reference:distellipsoid.ellipsoid.Ellipsoid.numpoints}\pysiglinewithargsret{\sphinxbfcode{numpoints}}{}{}
Return the number of points.

\end{fulllineitems}

\index{plot() (distellipsoid.ellipsoid.Ellipsoid method)}

\begin{fulllineitems}
\phantomsection\label{api_reference:distellipsoid.ellipsoid.Ellipsoid.plot}\pysiglinewithargsret{\sphinxbfcode{plot}}{\emph{figure=None}, \emph{axes=None}, \emph{**kwargs}}{}
Plot graph of ellipsoid

\end{fulllineitems}

\index{plotsummary() (distellipsoid.ellipsoid.Ellipsoid method)}

\begin{fulllineitems}
\phantomsection\label{api_reference:distellipsoid.ellipsoid.Ellipsoid.plotsummary}\pysiglinewithargsret{\sphinxbfcode{plotsummary}}{\emph{**kwargs}}{}
Plot graph of ellipsoid with text of basic parameters

\end{fulllineitems}

\index{points (distellipsoid.ellipsoid.Ellipsoid attribute)}

\begin{fulllineitems}
\phantomsection\label{api_reference:distellipsoid.ellipsoid.Ellipsoid.points}\pysigline{\sphinxbfcode{points}}
Points to define a hyperellipse.

\end{fulllineitems}

\index{raderr() (distellipsoid.ellipsoid.Ellipsoid method)}

\begin{fulllineitems}
\phantomsection\label{api_reference:distellipsoid.ellipsoid.Ellipsoid.raderr}\pysiglinewithargsret{\sphinxbfcode{raderr}}{}{}
Return standard deviation in radii

\end{fulllineitems}

\index{radvar() (distellipsoid.ellipsoid.Ellipsoid method)}

\begin{fulllineitems}
\phantomsection\label{api_reference:distellipsoid.ellipsoid.Ellipsoid.radvar}\pysiglinewithargsret{\sphinxbfcode{radvar}}{}{}
Return variance of the radii.

\end{fulllineitems}

\index{shapeparam() (distellipsoid.ellipsoid.Ellipsoid method)}

\begin{fulllineitems}
\phantomsection\label{api_reference:distellipsoid.ellipsoid.Ellipsoid.shapeparam}\pysiglinewithargsret{\sphinxbfcode{shapeparam}}{}{}
Return ellipsoid shape measure r3/r2 - r2/r1.

\end{fulllineitems}

\index{shapeparam\_old() (distellipsoid.ellipsoid.Ellipsoid method)}

\begin{fulllineitems}
\phantomsection\label{api_reference:distellipsoid.ellipsoid.Ellipsoid.shapeparam_old}\pysiglinewithargsret{\sphinxbfcode{shapeparam\_old}}{}{}
Return ellipsoid shape measure r1/r2 - r2/r3.

\end{fulllineitems}

\index{sphererad() (distellipsoid.ellipsoid.Ellipsoid method)}

\begin{fulllineitems}
\phantomsection\label{api_reference:distellipsoid.ellipsoid.Ellipsoid.sphererad}\pysiglinewithargsret{\sphinxbfcode{sphererad}}{}{}
Return radius of sphere of equivalent volume as ellipsoid.

\end{fulllineitems}

\index{strainenergy() (distellipsoid.ellipsoid.Ellipsoid method)}

\begin{fulllineitems}
\phantomsection\label{api_reference:distellipsoid.ellipsoid.Ellipsoid.strainenergy}\pysiglinewithargsret{\sphinxbfcode{strainenergy}}{}{}
Return ellipsoid strain energy approximation.

\end{fulllineitems}

\index{uniquerad() (distellipsoid.ellipsoid.Ellipsoid method)}

\begin{fulllineitems}
\phantomsection\label{api_reference:distellipsoid.ellipsoid.Ellipsoid.uniquerad}\pysiglinewithargsret{\sphinxbfcode{uniquerad}}{\emph{tolerance=None}}{}
Determine the number of unique radii within tolerance (defaults to self.tolerance)

\end{fulllineitems}


\end{fulllineitems}



\section{Polyhedron}
\label{api_reference:polyhedron}
Represents the set of objects that define a coordination polyhedron, including transforming from a unit cell description
to one of orthogonal (Cartesian) positions.
\phantomsection\label{api_reference:module-distellipsoid.polyhedron}\index{distellipsoid.polyhedron (module)}
Module for containing polyhedron data and functions
\index{Polyhedron (class in distellipsoid.polyhedron)}

\begin{fulllineitems}
\phantomsection\label{api_reference:distellipsoid.polyhedron.Polyhedron}\pysiglinewithargsret{\sphinxstrong{class }\sphinxcode{distellipsoid.polyhedron.}\sphinxbfcode{Polyhedron}}{\emph{centre}, \emph{ligands}, \emph{atomdict=None}, \emph{ligtypes=None}}{}
Class to hold polyhedron object
\index{allbondlens() (distellipsoid.polyhedron.Polyhedron method)}

\begin{fulllineitems}
\phantomsection\label{api_reference:distellipsoid.polyhedron.Polyhedron.allbondlens}\pysiglinewithargsret{\sphinxbfcode{allbondlens}}{\emph{mtensor}}{}
Return bond lengths to all ligands

\end{fulllineitems}

\index{alldelabc() (distellipsoid.polyhedron.Polyhedron method)}

\begin{fulllineitems}
\phantomsection\label{api_reference:distellipsoid.polyhedron.Polyhedron.alldelabc}\pysiglinewithargsret{\sphinxbfcode{alldelabc}}{}{}
Return all coordinates relative to centre.

\end{fulllineitems}

\index{alldelxyz() (distellipsoid.polyhedron.Polyhedron method)}

\begin{fulllineitems}
\phantomsection\label{api_reference:distellipsoid.polyhedron.Polyhedron.alldelxyz}\pysiglinewithargsret{\sphinxbfcode{alldelxyz}}{\emph{orthom}}{}
Return all cartesian coordinates relative to centre.

\end{fulllineitems}

\index{allxyz() (distellipsoid.polyhedron.Polyhedron method)}

\begin{fulllineitems}
\phantomsection\label{api_reference:distellipsoid.polyhedron.Polyhedron.allxyz}\pysiglinewithargsret{\sphinxbfcode{allxyz}}{\emph{orthom}}{}
Return all atoms in cartesian coordinates.

\end{fulllineitems}

\index{averagebondlen() (distellipsoid.polyhedron.Polyhedron method)}

\begin{fulllineitems}
\phantomsection\label{api_reference:distellipsoid.polyhedron.Polyhedron.averagebondlen}\pysiglinewithargsret{\sphinxbfcode{averagebondlen}}{\emph{mtensor}}{}
Return average centre-ligand bond length

\end{fulllineitems}

\index{bondlensig() (distellipsoid.polyhedron.Polyhedron method)}

\begin{fulllineitems}
\phantomsection\label{api_reference:distellipsoid.polyhedron.Polyhedron.bondlensig}\pysiglinewithargsret{\sphinxbfcode{bondlensig}}{\emph{mtensor}}{}
Return standard deviation of bond lengths

\end{fulllineitems}

\index{bondlenvar() (distellipsoid.polyhedron.Polyhedron method)}

\begin{fulllineitems}
\phantomsection\label{api_reference:distellipsoid.polyhedron.Polyhedron.bondlenvar}\pysiglinewithargsret{\sphinxbfcode{bondlenvar}}{\emph{mtensor}}{}
Return variance of bond lengths

\end{fulllineitems}

\index{cenxyz() (distellipsoid.polyhedron.Polyhedron method)}

\begin{fulllineitems}
\phantomsection\label{api_reference:distellipsoid.polyhedron.Polyhedron.cenxyz}\pysiglinewithargsret{\sphinxbfcode{cenxyz}}{\emph{orthom}}{}
Return centre atom cartesian coordinates.

\end{fulllineitems}

\index{ligdelabc() (distellipsoid.polyhedron.Polyhedron method)}

\begin{fulllineitems}
\phantomsection\label{api_reference:distellipsoid.polyhedron.Polyhedron.ligdelabc}\pysiglinewithargsret{\sphinxbfcode{ligdelabc}}{}{}
Return ligand coordinates relative to centre.

\end{fulllineitems}

\index{ligdelxyz() (distellipsoid.polyhedron.Polyhedron method)}

\begin{fulllineitems}
\phantomsection\label{api_reference:distellipsoid.polyhedron.Polyhedron.ligdelxyz}\pysiglinewithargsret{\sphinxbfcode{ligdelxyz}}{\emph{orthom}}{}
Return ligand cartesian coordinates relative to centre.

\end{fulllineitems}

\index{ligxyz() (distellipsoid.polyhedron.Polyhedron method)}

\begin{fulllineitems}
\phantomsection\label{api_reference:distellipsoid.polyhedron.Polyhedron.ligxyz}\pysiglinewithargsret{\sphinxbfcode{ligxyz}}{\emph{orthom}}{}
Return ligand cartesian coordinates.

\end{fulllineitems}

\index{makeellipsoid() (distellipsoid.polyhedron.Polyhedron method)}

\begin{fulllineitems}
\phantomsection\label{api_reference:distellipsoid.polyhedron.Polyhedron.makeellipsoid}\pysiglinewithargsret{\sphinxbfcode{makeellipsoid}}{\emph{orthom}, \emph{**kwargs}}{}
Set up ellipsoid object and fit minimum bounding ellipsoid

\end{fulllineitems}

\index{pointcolours() (distellipsoid.polyhedron.Polyhedron method)}

\begin{fulllineitems}
\phantomsection\label{api_reference:distellipsoid.polyhedron.Polyhedron.pointcolours}\pysiglinewithargsret{\sphinxbfcode{pointcolours}}{}{}
Return a list of colours for points based on ligand type.

\end{fulllineitems}


\end{fulllineitems}



\section{Crystal}
\label{api_reference:crystal}
Class to hold a number of polyhedron objects, as well as unit cell parameters and orthogonalisation matrix, etc.
\index{Crystal (class in distellipsoid.readcoords)}

\begin{fulllineitems}
\phantomsection\label{api_reference:distellipsoid.readcoords.Crystal}\pysiglinewithargsret{\sphinxstrong{class }\sphinxcode{distellipsoid.readcoords.}\sphinxbfcode{Crystal}}{\emph{cell=None}, \emph{atoms=None}, \emph{atomtypes=None}}{}
Class to hold crystal data and resulting ellipsoids.
\index{makepolyhedron() (distellipsoid.readcoords.Crystal method)}

\begin{fulllineitems}
\phantomsection\label{api_reference:distellipsoid.readcoords.Crystal.makepolyhedron}\pysiglinewithargsret{\sphinxbfcode{makepolyhedron}}{\emph{centre}, \emph{ligands}, \emph{atomdict=None}, \emph{ligtypes=None}}{}
Make polyhedron from centre and ligands.

\end{fulllineitems}

\index{mtensor() (distellipsoid.readcoords.Crystal method)}

\begin{fulllineitems}
\phantomsection\label{api_reference:distellipsoid.readcoords.Crystal.mtensor}\pysiglinewithargsret{\sphinxbfcode{mtensor}}{}{}
Return metric tensor from cell parameters

\end{fulllineitems}

\index{orthomatrix() (distellipsoid.readcoords.Crystal method)}

\begin{fulllineitems}
\phantomsection\label{api_reference:distellipsoid.readcoords.Crystal.orthomatrix}\pysiglinewithargsret{\sphinxbfcode{orthomatrix}}{}{}
Return orthogonalisation matrix from cell parameters.

\end{fulllineitems}


\end{fulllineitems}



\section{Plot Ellipsoid}
\label{api_reference:plot-ellipsoid}
Class to generate 3D interactive images of ellipsoids.
\index{EllipsoidImage (class in distellipsoid.plotellipsoid)}

\begin{fulllineitems}
\phantomsection\label{api_reference:distellipsoid.plotellipsoid.EllipsoidImage}\pysiglinewithargsret{\sphinxstrong{class }\sphinxcode{distellipsoid.plotellipsoid.}\sphinxbfcode{EllipsoidImage}}{\emph{figure=None}, \emph{axes=None}}{}~\index{clearlast() (distellipsoid.plotellipsoid.EllipsoidImage method)}

\begin{fulllineitems}
\phantomsection\label{api_reference:distellipsoid.plotellipsoid.EllipsoidImage.clearlast}\pysiglinewithargsret{\sphinxbfcode{clearlast}}{\emph{removepoints=True}, \emph{removeaxes=True}, \emph{removeframe=True}}{}
Remove previous plot

\end{fulllineitems}

\index{plotell() (distellipsoid.plotellipsoid.EllipsoidImage method)}

\begin{fulllineitems}
\phantomsection\label{api_reference:distellipsoid.plotellipsoid.EllipsoidImage.plotell}\pysiglinewithargsret{\sphinxbfcode{plotell}}{\emph{ellipsoid}, \emph{plotpoints=True}, \emph{plotaxes=True}, \emph{cagecolor='b'}, \emph{axcols=None}, \emph{cagealpha=0.2}, \emph{pointcolor='r'}, \emph{pointmarker='o'}, \emph{pointscale=100}, \emph{title=None}, \emph{equalaxes=True}}{}
Plot an ellipsoid

\end{fulllineitems}


\end{fulllineitems}



\section{CIF calculation routines}
\label{api_reference:cif-calculation-routines}

\subsection{calcfromcif}
\label{api_reference:calcfromcif}\label{api_reference:id1}\index{calcfromcif() (in module distellipsoid.calcellipsoid)}

\begin{fulllineitems}
\phantomsection\label{api_reference:distellipsoid.calcellipsoid.calcfromcif}\pysiglinewithargsret{\sphinxcode{distellipsoid.calcellipsoid.}\sphinxbfcode{calcfromcif}}{\emph{CIF}, \emph{centres}, \emph{radius}, \emph{allligtypes={[}{]}}, \emph{alllignames={[}{]}}, \emph{**kwargs}}{}
Main routine for computing ellipsoids from CIF file.

\end{fulllineitems}



\subsection{multiCIF}
\label{api_reference:multicif}
The main module for computing ellipsoids from a number of files, using multiprocessing (one core per CIF file) if required. Largely contains routines for error checking
input commands and calling {\hyperref[api_reference:calcfromcif]{\sphinxcrossref{\DUrole{std,std-ref}{calcfromcif}}}}.
\phantomsection\label{api_reference:module-distellipsoid.multiCIF}\index{distellipsoid.multiCIF (module)}
Module for processing one or more CIF files using supplied options
To run from the command line, call the CIFellipsoid.py script.
\index{QueueHandler (class in distellipsoid.multiCIF)}

\begin{fulllineitems}
\phantomsection\label{api_reference:distellipsoid.multiCIF.QueueHandler}\pysiglinewithargsret{\sphinxstrong{class }\sphinxcode{distellipsoid.multiCIF.}\sphinxbfcode{QueueHandler}}{\emph{queue}}{}
This is a logging handler which sends events to a multiprocessing queue.
The plan is to add it to Python 3.2, but this can be copy pasted into
user code for use with earlier Python versions.
\index{emit() (distellipsoid.multiCIF.QueueHandler method)}

\begin{fulllineitems}
\phantomsection\label{api_reference:distellipsoid.multiCIF.QueueHandler.emit}\pysiglinewithargsret{\sphinxbfcode{emit}}{\emph{record}}{}
Emit a record.
Writes the LogRecord to the queue.

\end{fulllineitems}


\end{fulllineitems}

\index{check\_centres() (in module distellipsoid.multiCIF)}

\begin{fulllineitems}
\phantomsection\label{api_reference:distellipsoid.multiCIF.check_centres}\pysiglinewithargsret{\sphinxcode{distellipsoid.multiCIF.}\sphinxbfcode{check\_centres}}{\emph{cifs}, \emph{centres}}{}
Determine which centres to use based on CIF contents and user-supplied arguments.

\end{fulllineitems}

\index{check\_labels() (in module distellipsoid.multiCIF)}

\begin{fulllineitems}
\phantomsection\label{api_reference:distellipsoid.multiCIF.check_labels}\pysiglinewithargsret{\sphinxcode{distellipsoid.multiCIF.}\sphinxbfcode{check\_labels}}{\emph{labels}, \emph{alllabs}}{}
Check that all labels are present in all cif files, handling regular expressions if needed.
Returns
-------
list of labels to test,
list of labels to omit,
list of missing labels

\end{fulllineitems}

\index{check\_ligands() (in module distellipsoid.multiCIF)}

\begin{fulllineitems}
\phantomsection\label{api_reference:distellipsoid.multiCIF.check_ligands}\pysiglinewithargsret{\sphinxcode{distellipsoid.multiCIF.}\sphinxbfcode{check\_ligands}}{\emph{cifs}, \emph{ligtypes}, \emph{liglbls}}{}
Find lists of ligand labels and types that should be used to define ellipsoids based on supplied lists.
\begin{description}
\item[{NOTE: This function will find the appropriate combination of label/type commands to find all required ligands,}] \leavevmode
EXCEPT where a site is found in multiple CIFs with the same label, but different type. In this case, the
returned lists of labels and types \emph{should} work correctly when run through calcellipsoid.calcfromcif

\end{description}

\end{fulllineitems}

\index{listener\_configurer() (in module distellipsoid.multiCIF)}

\begin{fulllineitems}
\phantomsection\label{api_reference:distellipsoid.multiCIF.listener_configurer}\pysiglinewithargsret{\sphinxcode{distellipsoid.multiCIF.}\sphinxbfcode{listener\_configurer}}{\emph{name=None}}{}
Function to configure logging output from sub processes

\end{fulllineitems}

\index{listener\_empty\_config() (in module distellipsoid.multiCIF)}

\begin{fulllineitems}
\phantomsection\label{api_reference:distellipsoid.multiCIF.listener_empty_config}\pysiglinewithargsret{\sphinxcode{distellipsoid.multiCIF.}\sphinxbfcode{listener\_empty\_config}}{\emph{name=None}}{}
Empty listener configurer, to do nothing except pass logs directly to parent

\end{fulllineitems}

\index{listener\_process() (in module distellipsoid.multiCIF)}

\begin{fulllineitems}
\phantomsection\label{api_reference:distellipsoid.multiCIF.listener_process}\pysiglinewithargsret{\sphinxcode{distellipsoid.multiCIF.}\sphinxbfcode{listener\_process}}{\emph{queue}, \emph{configurer}}{}
Process waits for logging events on queue and handles then

\end{fulllineitems}

\index{main() (in module distellipsoid.multiCIF)}

\begin{fulllineitems}
\phantomsection\label{api_reference:distellipsoid.multiCIF.main}\pysiglinewithargsret{\sphinxcode{distellipsoid.multiCIF.}\sphinxbfcode{main}}{\emph{cifs}, \emph{centres}, \emph{**kwargs}}{}
Process all supplied cif files using options supplied as kwargs (or defaults).
\begin{description}
\item[{Returns}] \leavevmode\begin{description}
\item[{phases}] \leavevmode{[}Dict{]}
Dictionary of Crystal objects (containing ellipsoid results), keyed by CIF name.

\item[{plots}] \leavevmode{[}dict or dicts, optional{]}
Dictionary of summary plots (keyed by CIF name) containing Dictionary
of plots (keyed by ellipsoid centre)

\end{description}

\end{description}

\end{fulllineitems}

\index{run\_parallel() (in module distellipsoid.multiCIF)}

\begin{fulllineitems}
\phantomsection\label{api_reference:distellipsoid.multiCIF.run_parallel}\pysiglinewithargsret{\sphinxcode{distellipsoid.multiCIF.}\sphinxbfcode{run\_parallel}}{\emph{cifs}, \emph{testcen}, \emph{radius=3.0}, \emph{ligtypes={[}{]}}, \emph{lignames={[}{]}}, \emph{maxcycles=None}, \emph{tolerance=1e-06}, \emph{procs=None}}{}
Run ellipsoid computation in parallel, by CIF file

\end{fulllineitems}

\index{worker\_configure() (in module distellipsoid.multiCIF)}

\begin{fulllineitems}
\phantomsection\label{api_reference:distellipsoid.multiCIF.worker_configure}\pysiglinewithargsret{\sphinxcode{distellipsoid.multiCIF.}\sphinxbfcode{worker\_configure}}{\emph{queue}}{}
Initialise logging on worker

\end{fulllineitems}



\section{Utility Functions}
\label{api_reference:utility-functions}
A number of utility functions are provided to simplify generation of polyhedra and reading/writing of files.


\subsection{Unit Cell Functions}
\label{api_reference:unit-cell-functions}\index{makeP1cell() (in module distellipsoid.readcoords)}

\begin{fulllineitems}
\phantomsection\label{api_reference:distellipsoid.readcoords.makeP1cell}\pysiglinewithargsret{\sphinxcode{distellipsoid.readcoords.}\sphinxbfcode{makeP1cell}}{\emph{atomcoords}, \emph{symmops}, \emph{symmid}}{}
Generate full unit cell contents from symmetry operations from Cif file

Returned atom labels (as dict keys) are appended with `\_\#\#' to denote the number
of the symmetry operation that generated them from cif file (initial position
label is not changed).

\end{fulllineitems}

\index{findligands() (in module distellipsoid.readcoords)}

\begin{fulllineitems}
\phantomsection\label{api_reference:distellipsoid.readcoords.findligands}\pysiglinewithargsret{\sphinxcode{distellipsoid.readcoords.}\sphinxbfcode{findligands}}{\emph{centre}, \emph{atomcoords}, \emph{orthom}, \emph{radius=2.0}, \emph{types={[}{]}}, \emph{names={[}{]}}, \emph{atomtypes=None}}{}
Find all atoms within radius of centre

\end{fulllineitems}



\subsubsection{File Functions}
\label{api_reference:file-functions}\index{readcif() (in module distellipsoid.readcoords)}

\begin{fulllineitems}
\phantomsection\label{api_reference:distellipsoid.readcoords.readcif}\pysiglinewithargsret{\sphinxcode{distellipsoid.readcoords.}\sphinxbfcode{readcif}}{\emph{FILE}}{}
Read useful data from cif using PyCifRW.

\end{fulllineitems}

\phantomsection\label{api_reference:module-distellipsoid.writeproperties}\index{distellipsoid.writeproperties (module)}
Module to write distellipsoid ellipsoid parameters to text file.
\index{writeall() (in module distellipsoid.writeproperties)}

\begin{fulllineitems}
\phantomsection\label{api_reference:distellipsoid.writeproperties.writeall}\pysiglinewithargsret{\sphinxcode{distellipsoid.writeproperties.}\sphinxbfcode{writeall}}{\emph{FILE}, \emph{phase}, \emph{verbosity=3}, \emph{overwrite=False}}{}
Write data to file

Increasing verbosity above 0 will increase the amount of data printed to file
(4 is maximum output)

\end{fulllineitems}



\chapter{Glossary}
\label{glossary:glossary}\label{glossary:citation}\label{glossary::doc}\begin{description}
\item[{CIF\index{CIF|textbf}}] \leavevmode\phantomsection\label{glossary:term-cif}
\href{http://www.iucr.org/resources/cif}{Crystallographic Information File}

\item[{ellipsoid\index{ellipsoid|textbf}}] \leavevmode\phantomsection\label{glossary:term-ellipsoid}
A three-dimensional object formed by rotating an ellipse.

\item[{citation\index{citation|textbf}}] \leavevmode\phantomsection\label{glossary:term-citation}
James Cumby \& J. Paul Attfield, \emph{Ellipsoidal Analysis of Coordination Polyhedra}.

\item[{MBE\index{MBE|textbf}}] \leavevmode\phantomsection\label{glossary:term-mbe}
Minimum Bounding Ellipsoid. The smallest volume {\hyperref[glossary:term\string-ellipsoid]{\sphinxtermref{\DUrole{xref,std,std-term}{ellipsoid}}}} that can surround a set of points.

\item[{GUI\index{GUI|textbf}}] \leavevmode\phantomsection\label{glossary:term-gui}
Graphical User Interface

\end{description}


\chapter{Indices and tables}
\label{index:indices-and-tables}\label{index:citation}\begin{itemize}
\item {} 
\DUrole{xref,std,std-ref}{genindex}

\item {} 
\DUrole{xref,std,std-ref}{modindex}

\item {} 
\DUrole{xref,std,std-ref}{search}

\end{itemize}


\renewcommand{\indexname}{Python Module Index}
\begin{theindex}
\def\bigletter#1{{\Large\sffamily#1}\nopagebreak\vspace{1mm}}
\bigletter{d}
\item {\texttt{distellipsoid.ellipsoid}}, \pageref{api_reference:module-distellipsoid.ellipsoid}
\item {\texttt{distellipsoid.multiCIF}}, \pageref{api_reference:module-distellipsoid.multiCIF}
\item {\texttt{distellipsoid.polyhedron}}, \pageref{api_reference:module-distellipsoid.polyhedron}
\item {\texttt{distellipsoid.writeproperties}}, \pageref{api_reference:module-distellipsoid.writeproperties}
\end{theindex}

\renewcommand{\indexname}{Index}
\printindex
\end{document}
